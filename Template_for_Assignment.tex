% Assignment template using exam class
\documentclass[12pt,addpoints]{exam}
\usepackage[UTF8]{ctex}
\usepackage{amsmath,amssymb}
\usepackage{graphicx}
\usepackage{booktabs}
\usepackage{multirow}
\usepackage{geometry}
% Slightly tighter, balanced margins for readability
\geometry{a4paper, left=2.5cm, right=2.5cm, top=2.5cm, bottom=2.5cm}
\usepackage{mdframed}

% Header/footer setup
\pagestyle{headandfoot}
\firstpageheadrule
\firstpageheader{Assignment 1: Title}{Course / Instructor}{Date}
\runningheadrule
\runningheader{Assignment 1: Title}{Your Name}{\thepage}
\runningfooter{}{}{}

% Boxed answer environment
\newenvironment{answerbox}{\begin{mdframed}[linewidth=0.5pt]}{\end{mdframed}}

\title{Assignment 1: Title}
\author{Name: xxxx \quad ID: xxxx}
\date{Date: 2026-1-19}

\begin{document}

\maketitle

\begin{questions}

\question[10] 设函数 $f(x) = x^3 - 3x + 1$。
\begin{parts}
    \part 求 $x=1$ 处的极值类型。
    \part 写出该点的切线方程。
\end{parts}

\begin{answerbox}
    \textbf{Solution:\\}
    (a) 计算导数 $f'(x) = 3x^2 - 3$,在 $x=1$ 处,$f'(1) = 0$。\\ 计算二阶导数 $f''(x) = 6x$,在 $x=1$ 处,$f''(1) = 6 > 0$,因此 $x=1$ 处有极小值。\\
    (b) 切线方程为 $y - f(1) = f'(1)(x - 1)$,即 $y - (-1) = 0(x - 1)$,化简得 $y = -1$。
    
\end{answerbox}

\question[10] Your second problem statement goes here.
\begin{parts}
    \part First part of the second problem.
    \part Second part of the second problem.
\end{parts}

\begin{answerbox}
    \textbf{Solution: \\ (a)} Solution to the first part.\\
    \textbf{(b)} Solution to the second part.
\end{answerbox}

\question[10] Your third problem statement goes here.

\begin{parts}
    \part First part of the third problem.
    \part Second part of the third problem.
\end{parts}

\begin{answerbox}
    \textbf{Solution: \\ (a)} Solution to the first part.\\
    \textbf{(b)} Solution to the second part.  
\end{answerbox}

\end{questions}

\end{document}
